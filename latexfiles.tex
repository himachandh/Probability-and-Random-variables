\documentclass[10pt]{article}
\begin{document}

\title{\begin{Huge}Assignment 1\end{Huge}\\AI1110:Probabolity and Random Variables}
\author{Hima chandh\\AI22BTECH11009}
\maketitle
\textbf{12.13.2.6}:Let E and F be events with $P(E)$=$\frac{3}{5}$,$P(F)$=$\frac{3}{10}$and $P(E and F)$=$\frac{1}{5}$.
Are E and F independent?

\textbf{Solution}:Given,

\begin{center}
$P(E)$=$\frac{3}{5}$
\\$P(F)$=$\frac{3}{10}$
\\$P(E and F)$=$\frac{1}{5}$
\end{center}

2Events are said to be independent iff the product of probabilities of occurence of the events is equals to the probability of occurence of both events.
\\i.e, E and F are independent iff,

\begin{center}
$P(E).P(F) = P(E and F)$
\end{center}

consider the product of P(E) and P(F)

\begin{center}
$P(E).P(F)$ = $\frac{3}{5}$ . $\frac{3}{10}$ 

$P(E).P(F)$ = $\frac{9}{50}$
\end{center}
W.K.T,
\begin{center}
P(E and F)=$\frac{1}{5}$
\end{center}
Since,
\begin{center}
$$P(E).P(F) \neq P(E and F)$$
\end{center}
E and F are not independent events.

\end{document}